%-------------------------
% Resume in Latex
% Author : Alexander Buyantuev
% Based off of: https://github.com/sb2nov/resume
% License : MIT
%------------------------

\documentclass[english,russian,letterpaper,11pt]{article}

\usepackage{latexsym}
\usepackage[empty]{fullpage}
\usepackage{titlesec}
\usepackage{marvosym}
\usepackage[usenames,dvipsnames]{color}
\usepackage{verbatim}
\usepackage{enumitem}
\usepackage[hidelinks]{hyperref}
\usepackage{fancyhdr}
\usepackage[english]{babel}
\usepackage{tabularx}
\usepackage[T2A]{fontenc}
\usepackage[utf8]{inputenc}
\usepackage{biblatex}
\usepackage{csquotes}
\usepackage{graphicx}
\usepackage{amsmath}
\usepackage{amssymb}
\usepackage{amsthm}
\usepackage{caption}
\usepackage{tikz}
\usepackage{subcaption}
\usepackage{imakeidx}
\usepackage{fontawesome}
\usepackage{xspace}
\usepackage[russian]{cleveref}
\usepackage[a4paper,left=15mm,right=15mm,top=30mm,bottom=20mm]{geometry}
\usepackage[cache=false]{minted}
\usepackage{wasysym}
\usepackage{dsfont}
\usepackage{tabularx}
\input{glyphtounicode}


%----------FONT OPTIONS----------
% sans-serif
% \usepackage[sfdefault]{FiraSans}
% \usepackage[sfdefault]{roboto}
% \usepackage[sfdefault]{noto-sans}
% \usepackage[default]{sourcesanspro}

% serif
% \usepackage{CormorantGaramond}
% \usepackage{charter}


\pagestyle{fancy}
\fancyhf{} % clear all header and footer fields
\fancyfoot{}
\renewcommand{\headrulewidth}{0pt}
\renewcommand{\footrulewidth}{0pt}

% Adjust margins
\addtolength{\oddsidemargin}{-0.5in}
\addtolength{\evensidemargin}{-0.5in}
\addtolength{\textwidth}{1in}
\addtolength{\topmargin}{-.5in}
\addtolength{\textheight}{1.0in}

\urlstyle{same}

\raggedbottom
\raggedright
\setlength{\tabcolsep}{0in}

% Sections formatting
\titleformat{\section}{
  \vspace{-4pt}\scshape\raggedright\large
}{}{0em}{}[\color{black}\titlerule \vspace{-5pt}]

% Ensure that generate pdf is machine readable/ATS parsable
\pdfgentounicode=1

%-------------------------
% Custom commands
\newcommand{\resumeItem}[1]{
  \item\small{
    {#1 \vspace{-2pt}}
  }
}

\newcommand{\resumeSubheading}[4]{
  \vspace{-2pt}\item
    \begin{tabular*}{0.97\textwidth}[t]{l@{\extracolsep{\fill}}r}
      \textbf{#1} & #2 \\
      \textit{\small#3} & \textit{\small #4} \\
    \end{tabular*}\vspace{-7pt}
}

\newcommand{\resumeWorkExpHeading}[4]{
  \vspace{-2pt}\item
    \begin{tabular*}{0.97\textwidth}[t]{l@{\extracolsep{\fill}}r}
      \textbf{#1} & \textit{#2} \\
      #3 & #4 \\
    \end{tabular*}\vspace{-7pt}
}

\newcommand{\resumeSubSubheading}[2]{
    \item
    \begin{tabular*}{0.97\textwidth}{l@{\extracolsep{\fill}}r}
      \textit{\small#1} & \textit{\small #2} \\
    \end{tabular*}\vspace{-7pt}
}

\newcommand{\resumeProjectHeading}[2]{
    \item
    \begin{tabular*}{0.97\textwidth}{l@{\extracolsep{\fill}}r}
      \small#1 & #2 \\
    \end{tabular*}\vspace{-7pt}
}

\newcommand{\resumeSubItem}[1]{\resumeItem{#1}\vspace{-4pt}}

\renewcommand\labelitemii{$\vcenter{\hbox{\tiny$\bullet$}}$}

\newcommand{\resumeSubHeadingListStart}{\begin{itemize}[leftmargin=0.15in, label={}]}
\newcommand{\resumeSubHeadingListEnd}{\end{itemize}}
\newcommand{\resumeItemListStart}{\begin{itemize}}
\newcommand{\resumeItemListEnd}{\end{itemize}\vspace{-5pt}}
\newcommand{\achievementsListStart}{\begin{itemize}}
\newcommand{\achievementsListEnd}{\end{itemize}\vspace{-8pt}}
\newcommand{\dutiesListStart}{\begin{itemize}}
\newcommand{\dutiesListEnd}{\end{itemize}\vspace{-10pt}}

%-------------------------------------------
%%%%%%  RESUME STARTS HERE  %%%%%%%%%%%%%%%%%%%%%%%%%%%%


\begin{document}

%----------HEADING----------
% \begin{tabular*}{\textwidth}{l@{\extracolsep{\fill}}r}
%   \textbf{\href{http://sourabhbajaj.com/}{\Large Sourabh Bajaj}} & Email : \href{mailto:sourabh@sourabhbajaj.com}{sourabh@sourabhbajaj.com}\\
%   \href{http://sourabhbajaj.com/}{http://www.sourabhbajaj.com} & Mobile : +1-123-456-7890 \\
% \end{tabular*}

\begin{center}
    \textbf{\Huge \scshape Александр Буянтуев} \\ \vspace{2pt}
    \small \faPhone \ +7 911 292 71 53 $|$ \href{mailto:alexbuyan.dev@gmail.com}{\faEnvelope \ \underline{alexbuyan.dev@gmail.com}} $|$ 
    \href{https://github.com/alexbuyan}{\faGithub \ \underline{alexbuyan}} $|$ \href{https://www.linkedin.com/in/alexander-buyantuev-063785223}{\faLinkedin \ \underline{alexbuyan}}
\end{center}


%-----------EDUCATION-----------
\section{Education}
  \resumeSubHeadingListStart
    \resumeSubheading
      {НИУ ВШЭ}{Санкт-Петербург, Россия}
      {Бакалавриат, специальность <<Прикладная математика и информатика>>}{Сен. 2020 - Июнь 2024}
      \resumeItemListStart
        \resumeItem{\textbf{Пройденные курс}: \textit{Алгоритмы и структуры данных, C/C++, Java, Python для веб-разработки, Python для бэкенд-разработки, Машинное обучение, Базы данных, Компьютерные сети, Математический анализ, Линейная алгебра, Дискретная математика, Теория вероятностей, Математическая статистика}}
        \resumeItem{\textbf{Курсы ШАД Яндекса}: \textit{Машинное обучение, Обработка естественного языка, Обучение с подкреплением}}
        \resumeItem{\textbf{Курсы AI Talent Hub}: \textit{Управление проектами в Data Science}}
      \resumeItemListEnd
  \resumeSubHeadingListEnd

%-----------EXPERIENCE-----------
\section{Опыт работы}
    \resumeSubHeadingListStart
    \resumeWorkExpHeading
      {ML Engineer}{Сен. 2023 - н.в.}
      {Huawei R\&D, Network Scheduling Team}{Санкт-Петербург, Россия}
    \resumeProjectHeading
          {\underline{\textbf{Построение избыточных кодов при помощи RL}} $|$ \emph{Python, PyTorch, FastAPI}}{Сен. 2023 - Июнь 2024}
          \resumeSubSubheading
            {Обязанности:}{}
          \dutiesListStart
            \resumeItem{Разработал ПО RL-FEC для построения избыточных кодов под разные условия и ограничения канала передачи}
            \resumeItem{Исследовал применение различных RL-алгоритмов (DDQN, SAC, PPO) для построения избыточных кодов}
            \resumeItem{Разработал FastAPI сервис для оценки эффективности избыточных кодов}
            \resumeItem{Сравнил полученные при помощи RL-FEC коды с подходами FlexFEC, LDPC и RS-кодами на BEC \\и модели Гильберта}
          \dutiesListEnd
          \resumeSubSubheading
            {Достижения:}{}
          \achievementsListStart
            \resumeItem{\textbf{Повысил точность} оценки эффективности избыточных кодов с $\mathbf{10^{-3}}$ до $\mathbf{10^{-6}}$}
            \resumeItem{\textbf{Ускорил} обучение агента в \textbf{50 раз}}
            \resumeItem{\textbf{Продемонстрировал преимущество} RL-FEC как \textbf{универсального} похода для построения избыточных кодов}
          \achievementsListEnd
    \resumeWorkExpHeading
      {Software Engineer Intern}{Нояб. 2022 - Сен. 2023}
      {Huawei R\&D, Cangjie Team}{Санкт-Петербург, Россия}
    \resumeProjectHeading
          {\underline{\textbf{Поддержка CSV файлов для Data-Driven тестирования в Cangjie}} $|$ \emph{Cangjie}}{Сен. 2023}
          \resumeSubSubheading
            {Обязанности:}{}
          \dutiesListStart
            \resumeItem{Разработал класс CsvParser на языке Cangjie для обработки данных из CSV файлов}
            \resumeItem{Разработал класс CsvStrategy для предоставления данных для модульных тестов}
          \dutiesListEnd
          \resumeSubSubheading
            {Достижения:}{}
          \achievementsListStart
            \resumeItem{\textbf{Добавил} классы в тестовый фреймворк Cangjie}
          \achievementsListEnd
    \resumeProjectHeading
          {\underline{\textbf{Декомпилятор LLVM IR для Cangjie}} $|$ \emph{C++, Python, GoogleTest}}{Nov. 2022 - June 2023}
          \resumeSubSubheading
            {Обязанности:}{}
          \dutiesListStart
            \resumeItem{Разработал инструмент, который восстанавливает пакеты, классы и функции исходного кода Cangjie в C-подобном формате из внутреннего представления LLVM IR}
            \resumeItem{Провел тестирование инструмента на более чем 300 открытых проектов на Сangjie}
            \resumeItem{Разработал систему параллельного тестирования, которая запускает 30 тестовых примеров по 100000 строк менее чем за минуту}
          \dutiesListEnd
          \resumeSubSubheading
            {Достижения:}{}
          \achievementsListStart
            \resumeItem{\textbf{Распространил} инструмент внутри команды Cangjie для анализа сгенерированного компилятором кода другими разработчиками}
            \resumeItem{\textbf{Уменьшил} временные ресурсы команды разработки, затрачиваемые на анализ порождаемого компилятором кода}
          \achievementsListEnd
    \resumeSubHeadingListEnd
%--------------------------------

%-----------PROJECTS-----------
\section{Проекты}
\resumeSubHeadingListStart
    \resumeProjectHeading
      {\href{https://github.com/alexbuyan/tg_x_rag}{\underline{\textbf{Телеграм бот с поддержкой RAG}} \faGithub} $|$ \emph{Python}}{Июнь 2024}
      \resumeItemListStart
        \resumeItem{Совместил \href{https://ollama.com/library/llama3}{\textit{Llama3:8b}} и \href{https://huggingface.co/sentence-transformers}{\textit{sentence-transformers}} эмбеддинги для создания RAG}
        \resumeItem{Использовал векторное хранилище \href{https://python.langchain.com/v0.1/docs/integrations/vectorstores/chroma/}{\textit{Chroma}} для хранения информации из пользовательских PDF файлов}
        \resumeItem{Разработал Телеграм бота для загрузки PDF файлов и взаимодействия со встроенной моделью}
      \resumeItemListEnd
    \resumeProjectHeading
      {\href{https://github.com/alexbuyan/vk-big-data/blob/hw2/notebooks/hw_spark.ipynb}{\underline{\textbf{Предсказание рейтингов фильмов внутри кластера Hadoop}} \faGithub} $|$ \emph{Python, Hadoop, Spark}}{Дек. 2024}
      \resumeItemListStart
        \resumeItem{Настроил Hadoop кластер и Spark сессию для работы с датасетами}
        \resumeItem{При помощи PySpark обработал датасеты для получения необходимой информации}
        \resumeItem{Обучил SGDRegressor для предсказания рейтинга фильма на основе тегов}
      \resumeItemListEnd
    \resumeProjectHeading
      {\href{https://github.com/alexbuyan/nlp_course/blob/2023/week01_embeddings/homework.ipynb}{\underline{\textbf{Машинный перевод с украинского на русский}} \faGithub} $|$ \emph{Python}}{Сен. 2023}
      \resumeItemListStart
        \resumeItem{Загрузил эмбеддинги для русского и украинского языков}
        \resumeItem{Построил пространственную карту эмбеддингов при помощи линейной регрессии}
        % \resumeItem{Increased results with orthogonal transformation SVD}
        \resumeItem{Реализовал машинный перевод с украинского на русский язык}
      \resumeItemListEnd
    \resumeProjectHeading
        {\href{https://huggingface.co/spaces/alexbuyan/yt_videos_comments_devops_projects}{\underline{\textbf{Генерация комментариев к YouTube видео}} \faGithub} $|$ \emph{Python}}{Апр. 2022}
        \resumeItemListStart
          \resumeItem{Обучил GPT-2 Large на датасете американских видео и комментариев}
          \resumeItem{Разработал приложение для взаимодействия с моделью при помощи PyTube и Gradio}
          \resumeItem{Загрузил проект на \href{https://huggingface.co/spaces/alexbuyan/yt_videos_comments_devops_projects}{\textit{Hugging Face}}}
        \resumeItemListEnd
    \resumeProjectHeading
        {\href{https://github.com/Pdf-Creator/pdf-editor}{\underline{\textbf{PDF редактор с поддержкой \LaTeX \xspace выражений}} \faGithub} $|$ \emph{Java}}{Мар. 2022 - Июнь 2022}
        \resumeItemListStart
          \resumeItem{Разработал конвертер UI объектов в PDF документ для переноса данных проекта в PDF файл}
          \resumeItem{Реализовал отрисовку \LaTeX \xspace выражений, которая позволяет пользователям работать с математическими формулами}
          \resumeItem{Разработал утилиту для загрузки и сохранения файлов, чтобы пользователи могли сохранять свои проекты}
          \resumeItem{Добавил поддержку шрифтов в UI и PDF для возможности кастомизации документов}
        \resumeItemListEnd
    \resumeProjectHeading
        {\href{https://github.com/alexbuyan/BPKproject}{\underline{\textbf{Мессенджер с досками Trello}} \faGithub} $|$ \emph{C++, PostgreSQL, Trello API}}{Янв. 2021 - Май 2021}
        \resumeItemListStart
          \resumeItem{Создал базу данных на PostgreSQL для хранения пользовательских данных}
          \resumeItem{Разработал обертку над библиотекой curl для взаимодействия с Trello API, чтобы добавить поддержку командных досок}
          \resumeItem{Разработал функционал сервера для обработки запросов к базе данных}
        \resumeItemListEnd
    \resumeProjectHeading
        {\href{https://github.com/Parser-Comparison/Parser-Comparison}{\underline{\textbf{Сравнение парсер-генераторов}} \faGithub} $|$ \emph{Python, Java, ANTLR4, Parglare}}{Окт. 2021}
        \resumeItemListStart
          \resumeItem{Исследовал функциональность и ограничения ANTLR4 и Parglare и провел сравнение c другими парсер-генераторами} \\
          \resumeItem{Сравнил производительность генераторов при разборе неоднозначной грамматики и собрал данные экспериментов для отчета} \\
          \resumeItem{Описал результаты исследования в отчете}
        \resumeItemListEnd
    % \resumeProjectHeading
    %     {\href{https://github.com/alexbuyan/ml_projects/tree/main/hybrid_strategy}{\underline{\textbf{Hybrid Strategy for Timeseries}} \faGithub} $|$ \emph{Python}}{Apr. 2022}
    %     \resumeItemListStart
    %       \resumeItem{Implemented hybrid strategy for timeseries from \href{https://www.kaggle.com/c/demand-forecasting-kernels-only}{\underline{\textbf{this dataset}}}}
    %       \resumeItem{The solution got \textbf{MSE 33.97}}
    %     \resumeItemListEnd
    % \resumeProjectHeading
    %     {\href{https://github.com/alexbuyan/ml_projects/tree/main/resnet18}{\underline{\textbf{ResNet18}} \faGithub} $|$ \emph{Python}}{Mar. 2021}
    %     \resumeItemListStart
    %       \resumeItem{Developed ResidualBlock, ResNetLayer, ResNet18 using PyTorch}
    %     \resumeItemListEnd
    % \resumeProjectHeading
    %     {\href{https://github.com/alexbuyan/ml_projects/tree/main/backprop}{\underline{\textbf{Backpropagation}} \faGithub} $|$ \emph{Python}}{Feb. 2022}
    %     \resumeItemListStart
    %       \resumeItem{Implemented backprop for \textbf{BatchNormalization}}
    %       \resumeItem{Implemented backprop for \textbf{Dropout}}
    %     \resumeItemListEnd
\resumeSubHeadingListEnd
%-------------------------------------------

%-----------PROGRAMMING SKILLS-----------
\section{Навыки}
 \begin{itemize}[leftmargin=0.15in, label={}]
    \small{\item{
     \textbf{Языки программирования}{: Python, C/C++, Java} \\
     \textbf{Технологии и фреймворки}{: git, PyTorch, StableBaselines, Gymnasium, Docker, FastAPI, LangChain, SQL } \\
     \textbf{Языки}{: Русский, Английский (C1)} \\
    }}
 \end{itemize}
%-------------------------------------------

%-----------Activities-----------
% \section{Other activities}
%  \begin{itemize}[leftmargin=0.15in, label={}]
%     \small{\item{
%      \textbf{Sport}{: Swimming, Cycling, Longboarding, Basketball} \\
%      \textbf{Hobby}{: Drone photography} \\ 
%      \textbf{Other}{: Counselor at the camp}
%     }}
%  \end{itemize}
%-------------------------------------------

\end{document}
