%-------------------------
% Resume in Latex
% Author : Alexander Buyantuev
% Based off of: https://github.com/sb2nov/resume
% License : MIT
%------------------------

\documentclass[english,russian,letterpaper,11pt]{article}

\usepackage{latexsym}
\usepackage[empty]{fullpage}
\usepackage{titlesec}
\usepackage{marvosym}
\usepackage[usenames,dvipsnames]{color}
\usepackage{verbatim}
\usepackage{enumitem}
\usepackage[hidelinks]{hyperref}
\usepackage{fancyhdr}
\usepackage[english]{babel}
\usepackage{tabularx}
\usepackage[T2A]{fontenc}
\usepackage[utf8]{inputenc}
\usepackage{biblatex}
\usepackage{csquotes}
\usepackage{graphicx}
\usepackage{amsmath}
\usepackage{amssymb}
\usepackage{amsthm}
\usepackage{caption}
\usepackage{tikz}
\usepackage{subcaption}
\usepackage{imakeidx}
\usepackage{fontawesome}
\usepackage{xspace}
\usepackage[russian]{cleveref}
\usepackage[a4paper,left=15mm,right=15mm,top=30mm,bottom=20mm]{geometry}
\usepackage[cache=false]{minted}
\usepackage{wasysym}
\usepackage{dsfont}
\usepackage{tabularx}
\input{glyphtounicode}


%----------FONT OPTIONS----------
% sans-serif
% \usepackage[sfdefault]{FiraSans}
% \usepackage[sfdefault]{roboto}
% \usepackage[sfdefault]{noto-sans}
% \usepackage[default]{sourcesanspro}

% serif
% \usepackage{CormorantGaramond}
% \usepackage{charter}


\pagestyle{fancy}
\fancyhf{} % clear all header and footer fields
\fancyfoot{}
\renewcommand{\headrulewidth}{0pt}
\renewcommand{\footrulewidth}{0pt}

% Adjust margins
\addtolength{\oddsidemargin}{-0.5in}
\addtolength{\evensidemargin}{-0.5in}
\addtolength{\textwidth}{1in}
\addtolength{\topmargin}{-.5in}
\addtolength{\textheight}{1.0in}

\urlstyle{same}

\raggedbottom
\raggedright
\setlength{\tabcolsep}{0in}

% Sections formatting
\titleformat{\section}{
  \vspace{-4pt}\scshape\raggedright\large
}{}{0em}{}[\color{black}\titlerule \vspace{-5pt}]

% Ensure that generate pdf is machine readable/ATS parsable
\pdfgentounicode=1

%-------------------------
% Custom commands
\newcommand{\resumeItem}[1]{
  \item\small{
    {#1 \vspace{-2pt}}
  }
}

\newcommand{\resumeSubheading}[4]{
  \vspace{-2pt}\item
    \begin{tabular*}{0.97\textwidth}[t]{l@{\extracolsep{\fill}}r}
      \textbf{#1} & #2 \\
      \textit{\small#3} & \textit{\small #4} \\
    \end{tabular*}\vspace{-7pt}
}

\newcommand{\resumeWorkExpHeading}[4]{
  \vspace{-2pt}\item
    \begin{tabular*}{0.97\textwidth}[t]{l@{\extracolsep{\fill}}r}
      \textbf{#1} & \textit{#2} \\
      #3 & #4 \\
    \end{tabular*}\vspace{-7pt}
}

\newcommand{\resumeSubSubheading}[2]{
    \item
    \begin{tabular*}{0.97\textwidth}{l@{\extracolsep{\fill}}r}
      \textit{\small#1} & \textit{\small #2} \\
    \end{tabular*}\vspace{-7pt}
}

\newcommand{\resumeProjectHeading}[2]{
    \item
    \begin{tabular*}{0.97\textwidth}{l@{\extracolsep{\fill}}r}
      \small#1 & #2 \\
    \end{tabular*}\vspace{-7pt}
}

\newcommand{\resumeSubItem}[1]{\resumeItem{#1}\vspace{-4pt}}

\renewcommand\labelitemii{$\vcenter{\hbox{\tiny$\bullet$}}$}

\newcommand{\resumeSubHeadingListStart}{\begin{itemize}[leftmargin=0.15in, label={}]}
\newcommand{\resumeSubHeadingListEnd}{\end{itemize}}
\newcommand{\resumeItemListStart}{\begin{itemize}}
\newcommand{\resumeItemListEnd}{\end{itemize}\vspace{-5pt}}

%-------------------------------------------
%%%%%%  RESUME STARTS HERE  %%%%%%%%%%%%%%%%%%%%%%%%%%%%


\begin{document}

%----------HEADING----------
% \begin{tabular*}{\textwidth}{l@{\extracolsep{\fill}}r}
%   \textbf{\href{http://sourabhbajaj.com/}{\Large Sourabh Bajaj}} & Email : \href{mailto:sourabh@sourabhbajaj.com}{sourabh@sourabhbajaj.com}\\
%   \href{http://sourabhbajaj.com/}{http://www.sourabhbajaj.com} & Mobile : +1-123-456-7890 \\
% \end{tabular*}

\begin{center}
    \textbf{\Huge \scshape Александр Буянтуев} \\ \vspace{2pt}
    \small \faPhone \ +7 911 292 71 53 $|$ \href{mailto:alexbuyan.dev@gmail.com}{\faEnvelope \ \underline{alexbuyan.dev@gmail.com}} $|$ 
    \href{https://github.com/alexbuyan}{\faGithub \ \underline{alexbuyan}} $|$ \href{https://www.linkedin.com/in/alexander-buyantuev-063785223}{\faLinkedin \ \underline{alexbuyan}}
\end{center}


%-----------EDUCATION-----------
\section{Образование}
  \resumeSubHeadingListStart
    \resumeSubheading
      {НИУ ВШЭ}{Санкт-Петербург, Россия}
      {Бакалавриат по специальности "Прикладная математика и информатика"}{Сент. 2020 - Авг. 2024}
      \resumeItemListStart
        \resumeItem{\textbf{Пройденные курсы}: \textit{Алгоритмы и Структуры данных, C/C++, Java, Python, Архитектура компьюьтера и операционные системы, Машинное обучение, База данных, Разработка ПО, Компьютерные сети, Математический анализ, Линейная алгебра, Дискретная математика, Теория вероятностей, Математическая статистика}}
      \resumeItemListEnd
  \resumeSubHeadingListEnd

%-----------PROGRAMMING SKILLS-----------
\section{Навыки}
 \begin{itemize}[leftmargin=0.15in, label={}]
    \small{\item{
     \textbf{Языки программирования}{: Java, C/C++, Python, Haskell, TypeScript} \\
     \textbf{Технологии и фреймворки}{: git, SQL, Docker, \LaTeX, Manim} \\
     \textbf{Языки}{: Русский, Английский (C1)} \\
    }}
 \end{itemize}
%-------------------------------------------

%-----------EXPERIENCE-----------
\section{Опыт работы}
    \resumeSubHeadingListStart
    \resumeWorkExpHeading
      {Стажер-разработчик}{Нояб. 2022 - н.в.}
      {Huawei R\&D, Cangjie Team}{Санкт-Петербург, Россия}
      \resumeProjectHeading
          {\underline{\textbf{Поддержка CSV файлов для Data-Driven тестирования в Cangjie}} $|$ \emph{Cangjie}}{Сент. 2023}
          \resumeItemListStart
            \resumeItem{Реализован CsvParser для Cangjie для \textbf{обработки данных} из CSV файлов}
            \resumeItem{Разработал CsvStrategy для предоставления данных для модульных тестов и \textbf{добавил} его в Cangjie Test Framework}
          \resumeItemListEnd
      \resumeProjectHeading
          {\underline{\textbf{Декомпилятор LLVM IR для Cangjie}} $|$ \emph{C++, Python, GoogleTest}}{Нояб. 2022 - Июнь 2023}
          \resumeItemListStart
            \resumeItem{Разработал инструмент для представления LLVM IR в C-подобном формате, который восстанавливает пакеты, классыы и функции из исходного кода Cangjie, для \textbf{ускорения} анализа порождаемого компилятором кода}
            \resumeItem{Реализовал обработчик LLVM GEP инстукций для отображения полей класса и их типов при обращения по указателю с целью \textbf{ускорения} читаемости кода}
            \resumeItem{Скачал исходный код из более чем 300 открытых проектов на Сangjie и создал из них тестовые примеры для \textbf{тестирования} моего инструмента}
            \resumeItem{Разработал систему параллельного тестирования, которая запускает 30 тестовых примеров по 100000 строк менее чем за минуту, чтобы \textbf{исправить ошибки} в моем инструменте}
            \resumeItem{\textbf{Распространил} инструмент внутри команды Cangjie для анализа сгенерированного компилятором кода другими разработчиками}
          \resumeItemListEnd
    \resumeSubHeadingListEnd
%--------------------------------

%-----------PROJECTS-----------
\section{Проекты}
    \resumeSubHeadingListStart
    \resumeProjectHeading
          {\href{https://github.com/Pdf-Creator/pdf-editor}{\underline{\textbf{PDF редактор с поддержкой \LaTeX \xspace выражений}} \faGithub} $|$ \emph{Java}}{Март. 2022 - Июнь 2022}
          \resumeItemListStart
            \resumeItem{Разработал конвертер UI объектов в PDF документ для \textbf{переноса} данных проекта в PDF файл}
            \resumeItem{Реализован \textbf{рендеринг} \LaTeX \xspace выражений, позволяющий пользователям работать с математическими формулами}
            \resumeItem{Разработана \textbf{утилита} для загрузки и сохранения файлов, позволяющая пользователям хранить свои проекты}
            \resumeItem{Добавлена поддержка шрифтов в UI и PDF для возможности \textbf{кастомизации} проектов}
          \resumeItemListEnd
    \resumeProjectHeading
    {\href{https://github.com/alexbuyan/BPKproject}{\underline{\textbf{Мессенджер с досками Trello}} \faGithub} $|$ \emph{C++, PostgreSQL, Trello API}}{Янв. 2021 -- Май 2021}
    \resumeItemListStart
      \resumeItem{Создал базу данных для \textbf{хранения} пользовательской информации}
      \resumeItem{Разработал обертку над библиотекой curl для взаимодействия с Trello API, чтобы добавить \textbf{поддержку} досок Trello}
      \resumeItem{Разработал функционал сервера по \textbf{обработке запросов} к базе данных}
    \resumeItemListEnd
    \resumeProjectHeading
        {\href{https://github.com/alexbuyan/VCS}{\underline{\textbf{Система контроля версий (VCS)}} \faGithub} $|$ \emph{Java}}{Май 2022}
        \resumeItemListStart
          \resumeItem{Реализовал систему контроля версий (VCS) с поддержкой основных операций git}
          \resumeItem{Разработал интерфейс командной строки для \textbf{взаимодействия} с VCS}
        \resumeItemListEnd
    \resumeProjectHeading
    {\href{https://github.com/Parser-Comparison/Parser-Comparison}{\underline{\textbf{Сравнение парсер-генераторов}} \faGithub} $|$ \emph{Python, Java, ANTLR4, Parglare}}{Окт. 2021}
    \resumeItemListStart
    \resumeItem{Исследовал функциональность и ограничения ANTLR4 и Parglare для \textbf{сравнения} c другими парсер-генераторами}
    \resumeItem{Сравнил производительность генераторов при разборе неоднозначной грамматики и \textbf{собрал} данные экспериментов}
    \resumeItem{\textbf{Описал} результаты исследования в отчете}
    \resumeItemListEnd
    \resumeProjectHeading
        {\href{https://github.com/alexbuyan/LambdaCalculator}{\underline{\textbf{Лямбда-калькулятор}} \faGithub} $|$ \emph{Haskell}}{Дек. 2021}
        \resumeItemListStart
          \resumeItem{Разработал библиотеку, которая позволяет $\beta$-редуцировать лямбда-термы и решать $\alpha$ и $\beta$ эквивалентности}
          \resumeItem{Разработал консольный парсер для \textbf{взаимодействия} с библиотекой}
        \resumeItemListEnd
    \resumeSubHeadingListEnd
%-------------------------------------------

%-----------Activities-----------
% \section{Other activities}
%  \begin{itemize}[leftmargin=0.15in, label={}]
%     \small{\item{
%      \textbf{Sport}{: Swimming, Cycling, Longboarding, Basketball} \\
%      \textbf{Hobby}{: Drone photography} \\ 
%      \textbf{Other}{: Counselor at the camp}
%     }}
%  \end{itemize}
%-------------------------------------------

\end{document}
